\chapter{Introduction}

\section{Revision history}
\begin{table}[!h]
\centering
\caption{Revision history}
\label{my-label}
\begin{tabular}{|c|c|}
\hline
Version & Date       \\ \hline
v1.0    & 15/01/2017 \\ \hline
\end{tabular}
\end{table}

\section{Purpose}
This document represents the Integration Test Plan (ITP) and its main purpose is to describe the tests that we need to perform in order to verify that all the components of the software, previously identified in the Design Document, are correctly linked and work together.
This document not only introduces the tests to be performed, but it also reasons the decision to take this set of tests, the order in which they will be executed and the expected result of each of them. In this step of the software design we consider components as black-boxes and the goal is to achieve a correct interoperability of them.

\section{Scope}
The aim of the project PowerEnjoy is to provide an automated service of car sharing.
After a registration, a client can hire a car near him/her through the web or mobile application and he/she can enjoy of all the extra services offered.
The exact position of the client is determined by the GPS signal of the client’s device or it’s allowed to manually insert a specific address.
So the system displays all the available cars in the client’s close area.
Then the client could make a reservation of a car, after which the system notifies the client with a message of confirmation with the car identifier.
If the reservation procedure successfully ends the chosen car won’t be available anymore for other clients.
Moreover a client cannot hire more than one car at the same time.
After the reservation the client has at most one hour to reach the car, when this time expires the system gives a penalty to the client and the car, previously hired, is available again for other clients.
The system allows the client to cancel his/her reservation.
When the client reaches the car, he/she can tell the system that is nearby through a specific button in the application and he/she starts to pay as soon as the engine ignites.
During the travel the system supervises the current charge of the car and notifies it to the client through a screen located in the car.
The system stops charging the amount of money that the client has to pay when he/she communicates through the application his/her decision to stop the rent.
When the car is parked in a safe area and the client exits, the system locks the car automatically and starts the procedure of payment.
The client is notified with the result of this procedure through an SMS, including the final fare.


\section{Definition, acronyms, abbrevations}

\subsection{Definition}

\begin{itemize}
\item \textbf{Guest client:} a person that is not already registered in the system or that has to log in.
\item \textbf{Registered client:} a person who has valid access credentials to log in the system.
\item \textbf{System administrator:} privileged user, in charge of managing administration processes and of updating business logic.
\item \textbf{Reservation:} it is the action performed by a registered client that allow him/her to reserve an available car for maximum one hour.
\item \textbf{Journey time = travel time:} time elapsed since the user starts the engine to the user parks the car and terminates the journey.
\item \textbf{Available car:} a car that is not reserved by any user and has enough charge to be rented.
\item \textbf{Unavailable car:} a car that is already reserved or damaged, so impossible to reserve.
\item \textbf{Gps navigation:} it is the navigation system that is included in the car on board system. It could be used by the user to find direction to the final destination.
\item \textbf{Final destination:} address where the user wants to go.
\item \textbf{Safe area:} the region where is permitted to park and leave a car once the rent is terminated.
\item \textbf{Power grid station:} the area where it’s allowed users to park the cars, leaving them attached to the power grid.
\end{itemize}

\subsection{Acronyms}

\begin{itemize}
\item \textbf{ITP: } Integration Test Plan
\item \textbf{DD: } Design Document
\item \textbf{RASD:} Requirements Analysis and Specification Document
\item \textbf{API:} Application Programming Interface
\item \textbf{UI:} User Interface
\item \textbf{DBMS: } Data Base management system
\end{itemize}

\subsection{Abbrevetations}

\begin{itemize}
\item \textbf{{[In.m]}:} n: level of integration. m: test number.
\end{itemize}

\section{Reference documents}

\begin{itemize}
\item RASD v1.1
\item DD v1.0
\item PowerEnjoy specification document (assignment).
\item IEEE Std 1016tm-2009 Standard for Information Technology - System Design - Software Design Descriptions.
\end{itemize}
\section{Document overview}
The ITP is composed of five sections:
\begin{itemize}
\item Introduction: it provides a general description of the content of the Integration Plan Document and some additional information in order to be coherent with the RASD document and the Design Document, previously produced.
\item Integration Strategy: this section introduces the strategy used for the integration, motivates the choices made and explains also the approach used to integrate. In fact the sequence to follow to perform the integration tests is specified.
\item Individual Steps and Test Description: in this part all the test sets are defined and described according to the choices made in the previous section. In particular the expected result of each test case is specified.
\item Tools and Test Equipment Required: this section contains a list of all the software tools and test equipment needed to perform the integration tests.
\item Program Stubs and Test Data Required: in the last part of the document all the program stubs and all the special test data required for the integration test phase are identified and characterized.
\end{itemize}
