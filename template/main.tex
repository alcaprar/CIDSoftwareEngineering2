\documentclass[12pt]{report}
\usepackage[english]{babel}
\usepackage{natbib}
\usepackage{url}
\usepackage[utf8x]{inputenc}
\usepackage{amsmath}
\usepackage{graphicx}
\graphicspath{{images/}}
\usepackage{parskip}
\usepackage{fancyhdr}
\usepackage{vmargin}
\setmarginsrb{3 cm}{2.5 cm}{3 cm}{2.5 cm}{1 cm}{1.5 cm}{1 cm}{1.5 cm}

\title{PowerEnjoy\\RASD}                             % Title
\date{\today}                                           % Date

\makeatletter
\let\thetitle\@title
\let\thedate\@date
\makeatother

\pagestyle{fancy}
\fancyhf{}
\lhead{\thetitle}
\cfoot{\thepage}

\begin{document}

%%%%%%%%%%%%%%%%%%%%%%%%%%%%%%%%%%%%%%%%%%%%%%%%%%%%%%%%%%%%%%%%%%%%%%%%%%%%%%%%%%%%%%%%%

\begin{titlepage}
    \centering
    \vspace*{0.5 cm}
    \includegraphics[scale = 0.75]{images/poli.jpg}\\[1.0 cm]   % University Logo
    \textsc{\LARGE Politecnico di Milano}\\[2.0 cm]   % University Name
    \textsc{\large Software engineering 2}\\[0.5 cm]               % Course Name
    \rule{\linewidth}{0.2 mm} \\[0.4 cm]
    { \huge \bfseries \thetitle}\\
    \rule{\linewidth}{0.2 mm} \\[1.5 cm]
    
    \begin{minipage}{0.4\textwidth}
        \begin{flushleft} \large
            Alessandro Caprarelli \\
	 Roberta Iero \\
	 Giorgio De Luca
	
            \end{flushleft}
            \end{minipage}~
            \begin{minipage}{0.4\textwidth}
            \begin{flushright} \large
            XXXXXX000 \\
            XXXXXX000 \\
            XXXXXX000                                   
        \end{flushright}
    \end{minipage}\\[2 cm]
    
    {\large \thedate}\\[2 cm]
 
    \vfill
    
\end{titlepage}

%%%%%%%%%%%%%%%%%%%%%%%%%%%%%%%%%%%%%%%%%%%%%%%%%%%%%%%%%%%%%%%%%%%%%%%%%%%%%%%%%%%%%%%%%

\tableofcontents
\pagebreak

%%%%%%%%%%%%%%%%%%%%%%%%%%%%%%%%%%%%%%%%%%%%%%%%%%%%%%%%%%%%%%%%%%%%%%%%%%%%%%%%%%%%%%%%%
\chapter{Chapter 1}
\section{About this design}
This is a simple report template with the UCT logo. Feel free to use/modify it to suit your needs. Variables that need to be altered have been commented to make modifications easier. For example if you need to change the university logo, look for the comment \texttt{\% University Logo} in this file and then make appropriate modifications in that line.

\subsection{subsection}

\subsubsection{subsubsection}
I hope that you find this template both visually appealing and useful. \\
\chapter{Introduction}

\section{Purpose}
The purpose of this document is to recap all the results deriving from the analysis of the code.
This analysis is performed taking into account the main inspection techniques that check if the code is ‘well-written’ or not.
The term ‘well-written’ means that the code has to be written following a certain set of rules.
These are summarized in the following checklist:
\begin{itemize}
\item Naming Conventions
\item Indention
\item Braces
\item File Organization
\item Wrapping Lines
\item Comments
\item Java Source Files
\item Package and Import Statements
\item Class and Interface Declarations
\item Initialization and Declarations
\item Method Calls
\item Arrays
\item Object Comparison
\item Output Format
\item Computation, Comparisons and Assignments
\item Exceptions
\item Flow of Control
\item Files
\end{itemize}

\section{Scope}
The main scope of this document is to give developers a list of mistakes to repair in order to make the code more robust and of quality.
In this way if the developers write the code following the same conventions, it will be also more readable.

\section{Definition, acronyms, abbrevations}

\subsection{Definition}

\subsection{Acronyms}

\begin{itemize}
\item \textbf{CI:} Code inspection
\end{itemize}

\subsection{Abbrevetations}

\section{Reference documents}

\begin{itemize}
\item Code inspection assignment document
\end{itemize}

\section{Document overview}
This document is composed of five sections:
\begin{itemize}
\item \textbf{Introduction:} this section contains the description of the document, of its purpose and some general information.
\item \textbf{Assigned classes:} this section contains the list of the classes that will be inspected in section 4.
\item \textbf{Functional role:} this part describes what the classes, that are going to be inspected in section 4, do.
\item \textbf{List of Issues:} in this last section are listed all the issues found during the inspection of the code of the previously described classes. In particular for each class is specified which kind of rule is violated and what would be the solution.
\end{itemize}


\hspace{1 cm}--- Linus


\end{document}
