\chapter{Introduction}

\section{Purpose}
This document represents the Design Document (DD).
The main purpose is to describe the system in terms of its components, providing a detailed description of the high level architecture, the design patterns we are going to use, the analysis of the internal and external interactions of the components.
This document is addressed to developers that have to implement the software.

\section{Scope}
The aim of the project PowerEnJoy is to provide an automated service of car sharing.
After a registration, a client can hire a car near him/her through the web or mobile application and he/she can enjoy of all the extra services offered.
The exact position of the client is determined by the GPS signal of the client’s device or it’s allowed to manually insert a specific address.
So the system displays all the available cars in the client’s close area.
Then the client could make a reservation of a car, after which the system notifies the client with a message of confirmation with the car identifier.
If the reservation procedure successfully ends the chosen car won’t be available anymore for other clients.
Moreover a client cannot hire more than one car at the same time.
After the reservation the client has at most one hour to reach the car, when this time expires the system gives a penalty to the client and the car, previously hired, is available again for other clients.
The system allows the client to cancel his/her reservation.
When the client reaches the car, he/she can tell the system that is nearby through a specific button in the application and he/she starts to pay as soon as the engine ignites.
During the travel the system supervises the current charge of the car and notifies it to the client through a screen located in the car.
The system stops charging the amount of money that the client has to pay when he/she communicates through the application his/her decision to stop the rent.
When the car is parked in a safe area and the client exits, the system locks the car automatically and starts the procedure of payment.
The client is notified with the result of this procedure through an SMS, including the final fare.

\section{Definition, acronyms, abbrevations}

\subsection{Definition}

\begin{itemize}
\item \textbf{Guest client:} a person that is not already registered in the system or that has to log in.
\item \textbf{Registered client:} a person who has valid access credentials to log in the system.
\item \textbf{System administrator:} privileged user, in charge of managing administration processes and of updating business logic.
\item \textbf{Reservation:} it is the action performed by a registered client that allow him/her to reserve an available car for maximum one hour.
\item \textbf{Journey time = travel time:} time elapsed since the user starts the engine to the user parks the car and terminates the journey.
\item \textbf{Available car:} a car that is not reserved by any user and has enough charge to be rented.
\item \textbf{Unavailable car:} a car that is already reserved or damaged, so impossible to reserve.
\item \textbf{Gps navigation:} it is the navigation system that is included in the car on board system. It could be used by the user to find direction to the final destination.
\item \textbf{Final destination:} address where the user wants to go.
\item \textbf{Safe area:} the region where is permitted to park and leave a car once the rent is terminated.
\item \textbf{Power grid station:} the area where it’s allowed users to park the cars, leaving them attached to the power grid.
\end{itemize}

\subsection{Acronyms}

\begin{itemize}
\item \textbf{RASD:} Requirements Analysis and Specification Document
\item \textbf{DD: } Design document
\item \textbf{API:} Application Programming Interface
\item \textbf{UI:} User Interface
\item \textbf{MVC: } Model View Controller
\item \textbf{URL: } Uniform Resource Locator
\item \textbf{UXD: } User Experience Design
\item \textbf{BCE: } Boundary Control Entity
\item \textbf{SMS: } Short Message Service
\item \textbf{WSS: } Web Socket Secure
\item \textbf{HTTPS: } HyperText Transfer Protocol over Secure Socket Layer

\end{itemize}

\subsection{Abbrevetations}

\begin{itemize}
\item \textbf{{[Gn]}:} n-goal
\item \textbf{{[Rn]}:} n-functional requirement
\end{itemize}

\section{Reference documents}

\begin{itemize}
\item PowerEnjoy specification document (assignment).
\item RASD v1.1 .
\item IEEE Std 1016tm-2009 Standard for Information Technology - System Design - Software Design Descriptions.
\end{itemize}

\section{Document overview}
The document is substantially divided in 5 sections:
\begin{itemize}
\item Introduction: it provides a general description of the content of the Design Document and some additional information in order to be coherent with the RASD document, previously produced.
\item Architecture Design: in this section there are a detailed increasing description of the components of our application, of their architecture and of their internal and external interactions.
This section also explains the main design and architectural choices.
\item Algorithms Design: this section contains an explanation of the main algorithms that have to be implemented in the software.
Pieces of pseudo-code are included in order to give a clear view of those algorithms.
\item User Interface Design: this section includes mockups and user experience explained via UX and BCE diagrams.
\item Requirements Traceability: this section relates the decisions taken in the RASD to the design components introduced in this document.
\end{itemize}
