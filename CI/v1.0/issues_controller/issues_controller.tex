\chapter{Issues related to ControllerViewArtifactInfo class}

\section{Naming conventions}
\subsection{Wrong naming conventions}
The class attribute \method{module} is declared at line 36 as static final and therefore its name should be in uppercase.
\begin{lstlisting}[firstnumber=36, caption={Issue}]
public static final String module = ControllerViewArtifactInfo.class.getName();
\end{lstlisting}
\begin{lstlisting}[firstnumber=36, caption={Possible solution}]
public static final String MODULE = ControllerViewArtifactInfo.class.getName();
\end{lstlisting}

\subsection{Meaningless names}
The name of the variable \method{that}, declared at line 114, is meaningless.
\begin{lstlisting}[firstnumber=114, caption={Issue}]
ControllerViewArtifactInfo that = (ControllerViewArtifactInfo) obj;
\end{lstlisting}
\begin{lstlisting}[firstnumber=114, caption={Possible solution}]
ControllerViewArtifactInfo cvai = (ControllerViewArtifactInfo) obj;
\end{lstlisting}


\section{File organization}

\subsection{Lines length more than 80 characters}
The following lines exceed 80 characters but are still acceptable:

\begin{lstlisting}[firstnumber=36, caption={Line 36 acceptable violation of the rule}]
public static final String module = ControllerViewArtifactInfo.class.getName();
\end{lstlisting}
\noindent\makebox[\linewidth]{\rule{\linewidth}{0.4pt}}
\begin{lstlisting}[firstnumber=59, caption={Line 59 acceptable violation of the rule}]
// populate screenCalledByThisView and reverse in aif.allViewInfosReferringToScreen
\end{lstlisting}
\noindent\makebox[\linewidth]{\rule{\linewidth}{0.4pt}}
\begin{lstlisting}[firstnumber=84, caption={Line 84 acceptable violation of the rule}]
String location = UtilURL.getOfbizHomeRelativeLocation(this.controllerXmlUrl);
\end{lstlisting}
\noindent\makebox[\linewidth]{\rule{\linewidth}{0.4pt}}
\begin{lstlisting}[firstnumber=115, caption={Line 115 acceptable violation of the rule}]
return UtilObject.equalsHelper(this.controllerXmlUrl, that.controllerXmlUrl) &&
\end{lstlisting}
\noindent\makebox[\linewidth]{\rule{\linewidth}{0.4pt}}
\begin{lstlisting}[firstnumber=122, caption={Line 122 acceptable violation of the rule}]
public Set<ControllerRequestArtifactInfo> getRequestsThatThisViewIsResponseTo() {
\end{lstlisting}

The following lines exceed 80 characters and may be reformatted in a better way:
\begin{lstlisting}[firstnumber=60, caption={Line 60 violation of the rule}]
if ("screen".equals(this.viewInfoMap.type) || "screenfop".equals(this.viewInfoMap.type) ||
                "screentext".equals(this.viewInfoMap.type) || "screenxml".equals(this.viewInfoMap.type)) {
\end{lstlisting}
\begin{lstlisting}[firstnumber=60, caption={Line 60 possible solution}]
if ("screen".equals(this.viewInfoMap.type) ||
    "screenfop".equals(this.viewInfoMap.type) ||
    "screentext".equals(this.viewInfoMap.type) ||
    "screenxml".equals(this.viewInfoMap.type)) {
\end{lstlisting}

\subsection{Lines length more than 120 characters}
The following lines exceed 120 characters and must be reformatted in a better way:
\begin{lstlisting}[firstnumber=45, caption={Line 45 violation of the rule}]
public ControllerViewArtifactInfo(URL controllerXmlUrl, String viewUri, ArtifactInfoFactory aif) throws GeneralException {
\end{lstlisting}
\begin{lstlisting}[firstnumber=45, caption={Line 45 possible solution}]
public ControllerViewArtifactInfo(URL controllerXmlUrl,
                                  String viewUri,
                                  ArtifactInfoFactory aif)
    throws GeneralException {
\end{lstlisting}
\noindent\makebox[\linewidth]{\rule{\linewidth}{0.4pt}}
\begin{lstlisting}[firstnumber=53, caption={Line 53 violation of the rule}]
throw new GeneralException("Could not find Controller View [" + viewUri + "] at URL [" + controllerXmlUrl.toExternalForm() + "]");
\end{lstlisting}
\begin{lstlisting}[firstnumber=53, caption={Line 53 possible solution}]
throw new GeneralException("Could not find Controller View [" + viewUri + "]" +
" at URL [" + controllerXmlUrl.toExternalForm() + "]");
\end{lstlisting}
\noindent\makebox[\linewidth]{\rule{\linewidth}{0.4pt}}
\begin{lstlisting}[firstnumber=57, caption={Line 57 violation of the rule}]
throw new GeneralException("Controller view with name [" + viewUri + "] is not defined in controller file [" + controllerXmlUrl + "].");
\end{lstlisting}
\begin{lstlisting}[firstnumber=57, caption={Line 57 possible solution}]
throw new GeneralException("Controller view with name [" + viewUri + "]" +
" is not defined in controller file [" + controllerXmlUrl + "].");
\end{lstlisting}
\noindent\makebox[\linewidth]{\rule{\linewidth}{0.4pt}}
\begin{lstlisting}[firstnumber=65, caption={Line 65 violation of the rule}]
this.screenCalledByThisView = this.aif.getScreenWidgetArtifactInfo(fullScreenName.substring(poundIndex+1), fullScreenName.substring(0, poundIndex));
\end{lstlisting}
\begin{lstlisting}[firstnumber=65, caption={Line 65 possible solution}]
this.screenCalledByThisView = this.aif.getScreenWidgetArtifactInfo
    (fullScreenName.substring(poundIndex+1), fullScreenName.substring(0, poundIndex));
\end{lstlisting}
\noindent\makebox[\linewidth]{\rule{\linewidth}{0.4pt}}
\begin{lstlisting}[firstnumber=68, caption={Line 68 violation of the rule}]
UtilMisc.addToSortedSetInMap(this, aif.allViewInfosReferringToScreen, this.screenCalledByThisView.getUniqueId());
\end{lstlisting}
\begin{lstlisting}[firstnumber=68, caption={Line 68 possible solution}]
UtilMisc.addToSortedSetInMap(this,
                            aif.allViewInfosReferringToScreen,
                            this.screenCalledByThisView.getUniqueId());
\end{lstlisting}

\subsection{Unuseful blank lines to remove}
The following blank lines are not useful to separate declarations of variables:
\begin{lstlisting}[firstnumber=40, caption={Lines 40-45 violation of the rule}]
protected URL controllerXmlUrl;
protected String viewUri;

protected ConfigXMLReader.ViewMap viewInfoMap;

protected ScreenWidgetArtifactInfo screenCalledByThisView = null;
\end{lstlisting}
\begin{lstlisting}[firstnumber=47, caption={Lines 47-50 violation of the rule}]
this.controllerXmlUrl = controllerXmlUrl;
this.viewUri = viewUri;

this.viewInfoMap = aif.getControllerViewMap(controllerXmlUrl, viewUri);
\end{lstlisting}

\subsection{Useful blank lines to add}
The following lines need a blank line to separate sections of code:
\begin{lstlisting}[firstnumber=18, caption={Lines 18-19 violation of the rule}]
*/
package org.apache.ofbiz.webtools.artifactinfo;
\end{lstlisting}
\begin{lstlisting}[firstnumber=18, caption={Lines 18-19 possible solution}]
*/

package org.apache.ofbiz.webtools.artifactinfo;
\end{lstlisting}

\section{Package and import statements}
The following declaration could be improved modifying the import statement:
\begin{lstlisting}[firstnumber=41, caption={Line 41 issue}]
protected ConfigXMLReader.ViewMap viewInfoMap;
\end{lstlisting}

\begin{lstlisting}[firstnumber=30, caption={Line 41 possible solution}]
import org.apache.ofbiz.webapp.control.ConfigXMLReader.ViewMap;

/**
 *
 */
public class ControllerViewArtifactInfo extends ArtifactInfoBase {
    public static final String module = ControllerViewArtifactInfo.class.getName();

    protected URL controllerXmlUrl;
    protected String viewUri;

    protected ViewMap viewInfoMap;
\end{lstlisting}

\section{Comments}
\subsection{Classes, interfaces, methods not described}
The following classes and methods should be properly described using Javadoc comments:
\begin{lstlisting}[firstnumber=32, caption={ControllerViewArtifactInfo class description missing}]
/**
 *
 */
public class ControllerViewArtifactInfo extends ArtifactInfoBase {
\end{lstlisting}

\begin{lstlisting}[firstnumber=44, caption={Constructor description missing}]
public ControllerViewArtifactInfo(URL controllerXmlUrl, String viewUri, ArtifactInfoFactory aif) throws GeneralException {
\end{lstlisting}

\section{Java source files}
Javadoc descriptions are missing for all the classes and methods.

\section{Class and interface declarations}
\subsection{Methods grouping by functionality}
The following method is in the middle of a group of getter methods, grouped by a functionality:
\begin{lstlisting}[firstnumber=113, caption={equals method in the middle of getters}]
@Override
public boolean equals(Object obj) {
    if (obj instanceof ControllerViewArtifactInfo) {
        ControllerViewArtifactInfo that = (ControllerViewArtifactInfo) obj;
        return UtilObject.equalsHelper(this.controllerXmlUrl, that.controllerXmlUrl) &&
            UtilObject.equalsHelper(this.viewUri, that.viewUri);
    } else {
        return false;
    }
}
\end{lstlisting}

\section{Initialization and declarations}
The following string should be declared as \method{private static final} variable and used instead of the plain text.
\begin{lstlisting}[firstnumber=113, caption={Constant is missing}]
if (location.endsWith("/WEB-INF/controller.xml")) {
\end{lstlisting}
\begin{lstlisting}[firstnumber=112, caption={Possible solution},label{lst:constant}]
private static final WEB-INF-CONTROLLER = "/WEB-INF/controller.xml";
if (location.endsWith(WEB-INF-CONTROLLER)) {
\end{lstlisting}

\section{Method calls}

\subsection{Hard-coded values}
The following method should have as \method{endIndex} a parameter related to the string constant proposed in the listing 4.32 and not an hard-coded value.
\begin{lstlisting}[firstnumber=88, caption={substring invocation}]
location = location.substring(0, location.length() - 23);
\end{lstlisting}
\begin{lstlisting}[firstnumber=86, caption={substring invocation possible solution}]
private static final WEB-INF-CONTROLLER = "/WEB-INF/controller.xml";
if (location.endsWith("/WEB-INF/controller.xml")) {
     location = location.substring(0, location.length() - WEB-INF-CONTROLLER.lenght);
}
\end{lstlisting}
\subsection{Method invocations on posssible null objects}
The following methods are invoked on objects that may be null therefore is needed a check:

\begin{lstlisting}[firstnumber=50, caption={getControllerViewMap invocation}]
this.viewInfoMap = aif.getControllerViewMap(controllerXmlUrl, viewUri);
\end{lstlisting}
\begin{lstlisting}[firstnumber=50, caption={getControllerViewMap invocation possible solution}]
if(aif!=null){
    this.viewInfoMap = aif.getControllerViewMap(controllerXmlUrl, viewUri);
}
\end{lstlisting}
\noindent\makebox[\linewidth]{\rule{\linewidth}{0.4pt}}

\begin{lstlisting}[firstnumber=87, caption={getControllerViewMap invocation}]
if (location.endsWith("/WEB-INF/controller.xml")) {
\end{lstlisting}
\begin{lstlisting}[firstnumber=87, caption={getControllerViewMap invocation possible solution}]
if(location!=null){
    if (location.endsWith("/WEB-INF/controller.xml")) {
}
\end{lstlisting}
\noindent\makebox[\linewidth]{\rule{\linewidth}{0.4pt}}

\section{Other errors}
The following second \method{if} statement is useless. It will never be executed.
\begin{lstlisting}[firstnumber=52, caption={Exeption would block the second if}]
if (this.viewInfoMap == null) {
    throw new GeneralException("Could not find Controller View [" + viewUri + "] at URL [" + controllerXmlUrl.toExternalForm() + "]");
}
if (this.viewInfoMap == null) {
    throw new GeneralException("Controller view with name [" + viewUri + "] is not defined in controller file [" + controllerXmlUrl + "].");
}
\end{lstlisting}


