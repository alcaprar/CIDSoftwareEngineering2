\chapter{Issues related to EbayStoreCategoryFacade class}

\section{Naming conventions}
\subsection{Wrong naming conventions}
The class attribute \method{module} is declared at line 58 as static final and therefore its name should be in uppercase.
\begin{lstlisting}[firstnumber=58, caption={Issue}]
public static final String module = EbayStoreCategoryFacade.class.getName();
\end{lstlisting}
\begin{lstlisting}[firstnumber=58, caption={Possible solution}]
public static final String Module = EbayStoreCategoryFacade.class.getName();
\end{lstlisting}
The method \method{AttributesEnabled()} at line 330, should start with a lower case character in order to match the camel cased pattern.
\begin{lstlisting}[firstnumber=330, caption={Issue}]
public boolean AttributesEnabled() {...}
\end{lstlisting}
\begin{lstlisting}[firstnumber=36, caption={Possible solution}]
public boolean attributesEnabled() {...}
\end{lstlisting}

\section{Braces}
\subsection{Curly braces missing}
The following \method{if} blocks should have curly braces to surround the only one statement.
\begin{lstlisting}[firstnumber=105, caption={Issue}]
if (recommendationsArray == null || recommendationsArray.length == 0)
	return;
\end{lstlisting}
\begin{lstlisting}[firstnumber=105, caption={Possible solution}]
if (recommendationsArray == null || recommendationsArray.length == 0){
    return;
}
\end{lstlisting}
\noindent\makebox[\linewidth]{\rule{\linewidth}{0.4pt}}
\begin{lstlisting}[firstnumber=271, caption={Issue}]
if (theme != null) temGroup.put("TemplateGroupName", theme.getGroupName());
\end{lstlisting}
\begin{lstlisting}[firstnumber=271, caption={Possible solution}]
if (theme != null) {
	temGroup.put("TemplateGroupName", theme.getGroupName());
}
\end{lstlisting}
\noindent\makebox[\linewidth]{\rule{\linewidth}{0.4pt}}
\begin{lstlisting}[firstnumber=274, caption={Issue}]
if (theme != null) temGroup.put("TemplateGroupName", "_NA_");
\end{lstlisting}
\begin{lstlisting}[firstnumber=274, caption={Possible solution}]
if (theme != null) {
	temGroup.put("TemplateGroupName", "_NA_");
}
\end{lstlisting}


\section{File organization}

\subsection{Lines length more than 80 characters}
The following lines exceed 80 characters but are still acceptable:

\begin{lstlisting}[firstnumber=71, caption={Line 71 acceptable violation of the rule}]
private StoreOwnerExtendedListingDurationsType storeOwnerExtendedListingDuration = null;
\end{lstlisting}
\noindent\makebox[\linewidth]{\rule{\linewidth}{0.4pt}}
\begin{lstlisting}[firstnumber=93, caption={Line 93 acceptable violation of the rule}]
AttributeSet[] itemSpecAttrSets = attrMaster.getItemSpecificAttributeSetsForCategories(ids);
\end{lstlisting}
\noindent\makebox[\linewidth]{\rule{\linewidth}{0.4pt}}
\begin{lstlisting}[firstnumber=94, caption={Line 94 acceptable violation of the rule}]
AttributeSet[] siteWideAttrSets = attrMaster.getSiteWideAttributeSetsForCategories(ids);
\end{lstlisting}
\noindent\makebox[\linewidth]{\rule{\linewidth}{0.4pt}}
\begin{lstlisting}[firstnumber=100, caption={Line 100 acceptable violation of the rule}]
GetCategorySpecificsCall getCatSpe = new GetCategorySpecificsCall(apiContext);
\end{lstlisting}
\noindent\makebox[\linewidth]{\rule{\linewidth}{0.4pt}}
\begin{lstlisting}[firstnumber=102, caption={Line 102 acceptable violation of the rule}]
DetailLevelCodeType[] detailLevels = new DetailLevelCodeType[] {DetailLevelCodeType.RETURN ALL};
\end{lstlisting}
\noindent\makebox[\linewidth]{\rule{\linewidth}{0.4pt}}
\begin{lstlisting}[firstnumber=104, caption={Line 104 acceptable violation of the rule}]
RecommendationsType[] recommendationsArray = getCatSpe.getCategorySpecifics();
\end{lstlisting}
\noindent\makebox[\linewidth]{\rule{\linewidth}{0.4pt}}
\begin{lstlisting}[firstnumber=115, caption={Line 115acceptable violation of the rule}]
SiteDefaultsType siteDefaults = this.siteFacade. getSiteFeatureDefaultMap().get(apiContext.getSite());
\end{lstlisting}
\noindent\makebox[\linewidth]{\rule{\linewidth}{0.4pt}}
\begin{lstlisting}[firstnumber=130, caption={Line 130 acceptable violation of the rule}]
ListingDurationDefinitionsType listDuration = featureDefinition.getListingDurations();
\end{lstlisting}
\noindent\makebox[\linewidth]{\rule{\linewidth}{0.4pt}}
\begin{lstlisting}[firstnumber=131, caption={Line 131 acceptable violation of the rule}]
ListingDurationDefinitionType[] durationArray = listDuration.getListingDuration();
\end{lstlisting}
\noindent\makebox[\linewidth]{\rule{\linewidth}{0.4pt}}
\begin{lstlisting}[firstnumber=134, caption={Line 134 acceptable violation of the rule}]
listingDurationMap.put(durationArray[i].getDurationSetID(), durationArray[i].getDuration());
\end{lstlisting}
\noindent\makebox[\linewidth]{\rule{\linewidth}{0.4pt}}
\begin{lstlisting}[firstnumber=144, caption={Line 144 acceptable violation of the rule}]
listingDurationReferenceMap.put(listingDuration[i].getType().value(), listingDuration[i].getValue());
\end{lstlisting}
\noindent\makebox[\linewidth]{\rule{\linewidth}{0.4pt}}
\begin{lstlisting}[firstnumber=156, caption={Line 156 acceptable violation of the rule}]
storeOwnerExtendedListingDuration = siteDefaults.getStoreOwnerExtendedListingDurations();
\end{lstlisting}
\noindent\makebox[\linewidth]{\rule{\linewidth}{0.4pt}}
\begin{lstlisting}[firstnumber=162, caption={Line 162 acceptable violation of the rule}]
private static BuyerPaymentMethodCodeType[] fiterPaymentMethod(BuyerPaymentMethodCodeType[] paymentMethods) {
\end{lstlisting}
\noindent\makebox[\linewidth]{\rule{\linewidth}{0.4pt}}
\begin{lstlisting}[firstnumber=163, caption={Line 163 acceptable violation of the rule}]
List<BuyerPaymentMethodCodeType> al = new ArrayList<BuyerPaymentMethodCodeType>();
\end{lstlisting}
\noindent\makebox[\linewidth]{\rule{\linewidth}{0.4pt}}
\begin{lstlisting}[firstnumber=203, caption={Line 203 acceptable violation of the rule}]
// invoke the method specified by methodName and return the corresponding return value
\end{lstlisting}
\noindent\makebox[\linewidth]{\rule{\linewidth}{0.4pt}}
\begin{lstlisting}[firstnumber=208, caption={Line 208 acceptable violation of the rule}]
private Object invokeMethodByName(CategoryFeatureType cf, String methodName) {...}
\end{lstlisting}
\noindent\makebox[\linewidth]{\rule{\linewidth}{0.4pt}}
\begin{lstlisting}[firstnumber=224, caption={Line 224 acceptable violation of the rule}]
List<Map<String,Object>> temGroupList = new LinkedList<Map<String,Object>>();
\end{lstlisting}
\noindent\makebox[\linewidth]{\rule{\linewidth}{0.4pt}}
\begin{lstlisting}[firstnumber=226, caption={Line 226 acceptable violation of the rule}]
GetDescriptionTemplatesCall call = new GetDescriptionTemplatesCall(this.apiContext);
\end{lstlisting}
\noindent\makebox[\linewidth]{\rule{\linewidth}{0.4pt}}
\begin{lstlisting}[firstnumber=231, caption={Line 231 acceptable violation of the rule}]
DescriptionTemplateType[] descriptionTemplateTypeList = resp.getDescriptionTemplate();
\end{lstlisting}
\noindent\makebox[\linewidth]{\rule{\linewidth}{0.4pt}}
\begin{lstlisting}[firstnumber=232, caption={Line 232 acceptable violation of the rule}]
Debug.logInfo("layout of category "+ this.catId +":"+ resp.getLayoutTotal(), module);
\end{lstlisting}
\noindent\makebox[\linewidth]{\rule{\linewidth}{0.4pt}}
\begin{lstlisting}[firstnumber=233, caption={Line 233 acceptable violation of the rule}]
for (DescriptionTemplateType descTemplateType : descriptionTemplateTypeList) {...}
\end{lstlisting}
\noindent\makebox[\linewidth]{\rule{\linewidth}{0.4pt}}
\begin{lstlisting}[firstnumber=236, caption={Line 236 acceptable violation of the rule}]
if ("THEME".equals(String.valueOf(descTemplateType.getType()))) {...}
\end{lstlisting}
\noindent\makebox[\linewidth]{\rule{\linewidth}{0.4pt}}
\begin{lstlisting}[firstnumber=238, caption={Line 238 acceptable violation of the rule}]
template.put("TemplateId", String.valueOf(descTemplateType.getID()));
\end{lstlisting}
\noindent\makebox[\linewidth]{\rule{\linewidth}{0.4pt}}
\begin{lstlisting}[firstnumber=239, caption={Line 239 acceptable violation of the rule}]
template.put("TemplateImageURL", descTemplateType.getImageURL());
\end{lstlisting}
\noindent\makebox[\linewidth]{\rule{\linewidth}{0.4pt}}
\begin{lstlisting}[firstnumber=245, caption={Line 245 acceptable violation of the rule}]
if (temGroup.get("TemplateGroupId"). equals(descTemplateType.getGroupID().toString())) {...}
\end{lstlisting}
\noindent\makebox[\linewidth]{\rule{\linewidth}{0.4pt}}
\begin{lstlisting}[firstnumber=253, caption={Line 253 acceptable violation of the rule}]
templateGroup.put("TemplateGroupId", descTemplateType.getGroupID().toString());
\end{lstlisting}
\noindent\makebox[\linewidth]{\rule{\linewidth}{0.4pt}}
\begin{lstlisting}[firstnumber=259, caption={Line 259 acceptable violation of the rule}]
templateList = UtilGenerics.checkList(templateGroup.get("Templates"));
\end{lstlisting}
\noindent\makebox[\linewidth]{\rule{\linewidth}{0.4pt}}
\begin{lstlisting}[firstnumber=263, caption={Line 263 acceptable violation of the rule}]
else if ("Layout".equals(String.valueOf(descTemplateType.getType()))) {...}
\end{lstlisting}
\noindent\makebox[\linewidth]{\rule{\linewidth}{0.4pt}}
\begin{lstlisting}[firstnumber=270, caption={Line 270 acceptable violation of the rule}]
if (theme.getGroupID() == Integer.parseInt(temGroup.get("TemplateGroupId").toString())) {...}
\end{lstlisting}
\noindent\makebox[\linewidth]{\rule{\linewidth}{0.4pt}}
\begin{lstlisting}[firstnumber=334, caption={Line 334 acceptable violation of the rule}]
public StoreOwnerExtendedListingDurationsType getStoreOwnerExtendedListingDuration() {...}
\end{lstlisting}
\noindent\makebox[\linewidth]{\rule{\linewidth}{0.4pt}}

The following lines exceed 80 characters and may be reformatted in a better way:
\begin{lstlisting}[firstnumber=99, caption={Line 99 violation of the rule}]
private void syncNameRecommendationTypes() throws ApiException, SdkException, Exception {...}
\end{lstlisting}
\begin{lstlisting}[firstnumber=99, caption={Line 99 possible solution}]
private void syncNameRecommendationTypes()
	throws ApiException, SdkException, Exception {...}
\end{lstlisting}
\noindent\makebox[\linewidth]{\rule{\linewidth}{0.4pt}}
\begin{lstlisting}[firstnumber=181, caption={Line 181 violation of the rule}]
private Object getInheritProperty(String catId,String methodName,
            Map<String, CategoryType> categoriesCacheMap, Map<String, CategoryFeatureType> cfsMap) throws Exception {...}
\end{lstlisting}
\begin{lstlisting}[firstnumber=181, caption={Line 181 possible solution}]
private Object getInheritProperty(String catId,
        String methodName,
        Map<String, CategoryType> categoriesCacheMap,
        Map<String, CategoryFeatureType> cfsMap)
    throws Exception {
\end{lstlisting}
\noindent\makebox[\linewidth]{\rule{\linewidth}{0.4pt}}
\begin{lstlisting}[firstnumber=199, caption={Line 199 violation of the rule}]
return getInheritProperty(cat.getCategoryParentID(0), methodName, categoriesCacheMap, cfsMap);
\end{lstlisting}
\begin{lstlisting}[firstnumber=199, caption={Line 199 possible solution}]
return getInheritProperty ( cat.getCategoryParentID(0),
    methodName,
    categoriesCacheMap,
    cfsMap);
\end{lstlisting}
\noindent\makebox[\linewidth]{\rule{\linewidth}{0.4pt}}
\begin{lstlisting}[firstnumber=221, caption={Line 221 violation of the rule}]
public List<Map<String,Object>> syncAdItemTemplates() throws ApiException, SdkSoapException, SdkException {...}
\end{lstlisting}
\begin{lstlisting}[firstnumber=221, caption={Line 221 possible solution}]
public List<Map<String,Object>> syncAdItemTemplates()
	throws ApiException, SdkSoapException, SdkException {
\end{lstlisting}




\subsection{Lines length more than 120 characters}
The following lines exceed 120 characters and must be reformatted in a better way:
\begin{lstlisting}[firstnumber=75, caption={Line 75 violation of the rule}]
public EbayStoreCategoryFacade(String catId, ApiContext apiContext, IAttributesMaster attrMaster, EbayStoreSiteFacade siteFacade) throws SdkException, Exception {...}
\end{lstlisting}
\begin{lstlisting}[firstnumber=75, caption={Line 75 possible solution}]
public EbayStoreCategoryFacade(String catId,
        ApiContext apiContext,
        IAttributesMaster attrMaster,
        EbayStoreSiteFacade siteFacade)
    throws SdkException, Exception {
\end{lstlisting}
\noindent\makebox[\linewidth]{\rule{\linewidth}{0.4pt}}
\begin{lstlisting}[firstnumber=112, caption={Line 112 violation of the rule}]
Map<String, CategoryType> categoriesCacheMap = this.siteFacade.getSiteCategoriesMap().get(apiContext.getSite());
\end{lstlisting}
\noindent\makebox[\linewidth]{\rule{\linewidth}{0.4pt}}
\begin{lstlisting}[firstnumber=95, caption={Line 95 violation of the rule}]
AttributeSet[] joinedAttrSets = attrMaster.joinItemSpecificAndSiteWideAttributeSets(itemSpecAttrSets, siteWideAttrSets);
\end{lstlisting}
\noindent\makebox[\linewidth]{\rule{\linewidth}{0.4pt}}
\begin{lstlisting}[firstnumber=114, caption={Line 114 violation of the rule}]
Map<String, CategoryFeatureType> cfsMap = this.siteFacade.getSiteCategoriesFeaturesMap().get(apiContext.getSite());
\end{lstlisting}
\noindent\makebox[\linewidth]{\rule{\linewidth}{0.4pt}}
\begin{lstlisting}[firstnumber=116, caption={Line 116 violation of the rule}]
FeatureDefinitionsType featureDefinition = this.siteFacade.getSiteFeatureDefinitionsMap().get(apiContext.getSite());
\end{lstlisting}
\noindent\makebox[\linewidth]{\rule{\linewidth}{0.4pt}}
\begin{lstlisting}[firstnumber=119, caption={Line 119 violation of the rule}]
itemSpecificEnabled = (ItemSpecificsEnabledCodeType)getInheritProperty(catId, "getItemSpecificsEnabled", categoriesCacheMap, cfsMap);
\end{lstlisting}
\begin{lstlisting}[firstnumber=119, caption={Line 119 possible solution}]
itemSpecificEnabled = (ItemSpecificsEnabledCodeType)getInheritProperty(catId,
        "getItemSpecificsEnabled",
        categoriesCacheMap,
        cfsMap);
\end{lstlisting}
\noindent\makebox[\linewidth]{\rule{\linewidth}{0.4pt}}
\begin{lstlisting}[firstnumber=124, caption={Line 124 violation of the rule}]
retPolicyEnabled = (Boolean)getInheritProperty(catId, "isReturnPolicyEnabled", categoriesCacheMap, cfsMap);
\end{lstlisting}
\begin{lstlisting}[firstnumber=124, caption={Line 124 possible solution}]
retPolicyEnabled = (Boolean)getInheritProperty(catId,
        "isReturnPolicyEnabled",
        categoriesCacheMap,
        cfsMap);
\end{lstlisting}
\noindent\makebox[\linewidth]{\rule{\linewidth}{0.4pt}}
\begin{lstlisting}[firstnumber=138, caption={Line 138 violation of the rule}]
ListingDurationReferenceType[] listingDuration = (ListingDurationReferenceType[])getInheritProperty(catId, "getListingDuration", categoriesCacheMap, cfsMap);
\end{lstlisting}
\begin{lstlisting}[firstnumber=138, caption={Line 138 possible solution}]
ListingDurationReferenceType[] listingDuration = (ListingDurationReferenceType[])getInheritProperty(catId,
        "getListingDuration",
        categoriesCacheMap,
        cfsMap);
\end{lstlisting}
\noindent\makebox[\linewidth]{\rule{\linewidth}{0.4pt}}
\begin{lstlisting}[firstnumber=138, caption={Line 138 violation of the rule}]
paymentMethods = (BuyerPaymentMethodCodeType[])getInheritProperty(catId, "getPaymentMethod", categoriesCacheMap, cfsMap);
\end{lstlisting}
\begin{lstlisting}[firstnumber=138, caption={Line 138 possible solution}]
paymentMethods = (BuyerPaymentMethodCodeType[])getInheritProperty(catId,
        "getPaymentMethod",
        categoriesCacheMap,
        cfsMap);
\end{lstlisting}


\subsection{Useless blank lines to remove}
The following blank lines are not useful to separate declarations of variables:
\begin{lstlisting}[firstnumber=63, caption={Lines 63 violation of the rule}]
private EbayStoreSiteFacade siteFacade = null;

private AttributeSet[] joinedAttrSets = null;
\end{lstlisting}
\begin{lstlisting}[firstnumber=113, caption={Lines 113 violation of the rule}]
Map<String, CategoryType> categoriesCacheMap = this.siteFacade.getSiteCategoriesMap().get(apiContext.getSite());

Map<String, CategoryFeatureType> cfsMap = this.siteFacade.getSiteCategoriesFeaturesMap().get(apiContext.getSite());
\end{lstlisting}
\begin{lstlisting}[firstnumber=155, caption={Lines 155 violation of the rule}]
paymentMethods = fiterPaymentMethod(paymentMethods);

storeOwnerExtendedListingDuration = siteDefaults.getStoreOwnerExtendedListingDurations();
\end{lstlisting}
\begin{lstlisting}[firstnumber=155, caption={Lines 155 violation of the rule}]
storeOwnerExtendedListingDuration = siteDefaults.getStoreOwnerExtendedListingDurations();

bestOfferEnabled = featureDefinition.getBestOfferEnabled();
\end{lstlisting}



\subsection{Useful blank lines to add}
The following lines need a blank line to separate sections of code:
\begin{lstlisting}[firstnumber=18, caption={Lines 18-19 violation of the rule}]
*/
package org.apache.ofbiz.ebaystore;
\end{lstlisting}
\begin{lstlisting}[firstnumber=18, caption={Lines 18-19 possible solution}]
*/

package org.apache.ofbiz.ebaystore;
\end{lstlisting}
\noindent\makebox[\linewidth]{\rule{\linewidth}{0.4pt}}
\begin{lstlisting}[firstnumber=37, caption={Lines 37-38 violation of the rule}]
import com.ebay.sdk.call.GetDescriptionTemplatesCall;
import com.ebay.soap.eBLBaseComponents.BestOfferEnabledDefinitionType;
\end{lstlisting}
\begin{lstlisting}[firstnumber=37, caption={Lines 37-38 possible solution}]
import com.ebay.sdk.call.GetDescriptionTemplatesCall;

import com.ebay.soap.eBLBaseComponents.BestOfferEnabledDefinitionType;
\end{lstlisting}
\noindent\makebox[\linewidth]{\rule{\linewidth}{0.4pt}}
\begin{lstlisting}[firstnumber=58, caption={Lines 58-59 violation of the rule}]
public static final String module = EbayStoreCategoryFacade.class.getName();
private ApiContext apiContext = null;
\end{lstlisting}
\begin{lstlisting}[firstnumber=58, caption={Lines 58-59 possible solution}]
public static final String module = EbayStoreCategoryFacade.class.getName();

private ApiContext apiContext = null;
\end{lstlisting}
\noindent\makebox[\linewidth]{\rule{\linewidth}{0.4pt}}
\begin{lstlisting}[firstnumber=122, caption={Lines 122-123 violation of the rule}]
}
//get returnPolicyEnabled feature
\end{lstlisting}
\begin{lstlisting}[firstnumber=122, caption={Lines 122-123 possible solution}]
}

//get returnPolicyEnabled feature
\end{lstlisting}
\noindent\makebox[\linewidth]{\rule{\linewidth}{0.4pt}}
\begin{lstlisting}[firstnumber=184, caption={Lines 184-185 violation of the rule}]
CategoryFeatureType cf = cfsMap.get(catId);
// invoke the method indicated by methodName
\end{lstlisting}
\begin{lstlisting}[firstnumber=184, caption={Lines 184-185 possible solution}]
CategoryFeatureType cf = cfsMap.get(catId);

// invoke the method indicated by methodName
\end{lstlisting}
\noindent\makebox[\linewidth]{\rule{\linewidth}{0.4pt}}
\begin{lstlisting}[firstnumber=229, caption={Lines 229-230 violation of the rule}]
 resp = (GetDescriptionTemplatesResponseType) call.execute(req);
 if (resp != null && "SUCCESS".equals(resp.getAck().toString())) {
\end{lstlisting}
\begin{lstlisting}[firstnumber=229, caption={Lines 229-230 possible solution}]
 resp = (GetDescriptionTemplatesResponseType) call.execute(req);

 if (resp != null && "SUCCESS".equals(resp.getAck().toString())) {
\end{lstlisting}
\noindent\makebox[\linewidth]{\rule{\linewidth}{0.4pt}}
\begin{lstlisting}[firstnumber=284, caption={Lines 284-285 violation of the rule}]
 List<Map<String,Object>> themes = new LinkedList<Map<String,Object>>();
 for (Map<String,Object> temp : this.adItemTemplates) {
\end{lstlisting}
\begin{lstlisting}[firstnumber=284, caption={Lines 284-285 possible solution}]
 List<Map<String,Object>> themes = new LinkedList<Map<String,Object>>();

 for (Map<String,Object> temp : this.adItemTemplates) {
\end{lstlisting}


\section{Comments}
\subsection{Classes, interfaces, methods not described}
The following classes and methods should be properly described using Javadoc comments:
\begin{lstlisting}[firstnumber=57, caption={EbayStoreCategoryFacade class Javadoc missing}]
public class EbayStoreCategoryFacade {...}
\end{lstlisting}

\begin{lstlisting}[firstnumber=74, caption={Constructor description missing}]
public EbayStoreCategoryFacade(String catId, ApiContext apiContext,
        IAttributesMaster attrMaster,
        EbayStoreSiteFacade siteFacade)
    throws SdkException, Exception {...}
\end{lstlisting}

\begin{lstlisting}[firstnumber=82, caption={Method description missing}]
private void syncCategoryMetaData() throws SdkException, Exception {...}
\end{lstlisting}
\begin{lstlisting}[firstnumber=89, caption={Method description missing}]
private void syncJoinedAttrSets() throws SdkException, Exception {...}
\end{lstlisting}
\begin{lstlisting}[firstnumber=98, caption={Method description missing}]
private void syncNameRecommendationTypes() throws ApiException, SdkException, Exception {...}
\end{lstlisting}
\begin{lstlisting}[firstnumber=110, caption={Method description missing}]
public void syncCategoryFeatures() throws Exception {...}
\end{lstlisting}
\begin{lstlisting}[firstnumber=220, caption={Method description missing}]
public List<Map<String,Object>> syncAdItemTemplates()
	throws ApiException, SdkSoapException, SdkException {...}
\end{lstlisting}
\begin{lstlisting}[firstnumber=329, caption={Method description missing}]
public boolean AttributesEnabled() {...}
\end{lstlisting}

\begin{lstlisting}[firstnumber=161, caption={method Javadoc missing}]
private static BuyerPaymentMethodCodeType[] fiterPaymentMethod(BuyerPaymentMethodCodeType[] paymentMethods) {...}
\end{lstlisting}

\subsection{Javadoc tags not described}
The following Javadoc tags should be more clear, specifying when a null value may be returned or an excpetion thrown.

\begin{lstlisting}[firstnumber=177, caption={tags not described}]
/**
* @return generic Object
* @throws Exception
*/
\end{lstlisting}

\begin{lstlisting}[firstnumber=205, caption={tags not described}]
/**
* @return generic object
*/
\end{lstlisting}

\subsection{Complex statement not described}
The following complex statements should be better documented to explain what they do.

\begin{lstlisting}[firstnumber=233, caption={complex for not documented}]
for (DescriptionTemplateType descTemplateType : descriptionTemplateTypeList) {
    List<Map<String,Object>> templateList = null;
    Map<String,Object> templateGroup = null;
    if ("THEME".equals(String.valueOf(descTemplateType.getType()))) {
        Map<String,Object> template = new HashMap<String, Object>();
        template.put("TemplateId", String.valueOf(descTemplateType.getID()));
        template.put("TemplateImageURL", descTemplateType.getImageURL());
        template.put("TemplateName", descTemplateType.getName());
        template.put("TemplateType", descTemplateType.getType());

        // check group template by groupId
        for (Map<String,Object> temGroup : temGroupList) {
            if (temGroup.get("TemplateGroupId").equals(descTemplateType.getGroupID().toString())) {
                templateGroup = temGroup;
                break;
            }
        }
        if (templateGroup == null) {
            templateGroup = new HashMap<String, Object>();
            templateList = new LinkedList<Map<String,Object>>();
            templateGroup.put("TemplateGroupId", descTemplateType.getGroupID().toString());
            templateList.add(template);
            templateGroup.put("Templates", templateList);
            temGroupList.add(templateGroup);
        } else {
            if (templateGroup.get("Templates") != null) {
                templateList = UtilGenerics.checkList(templateGroup.get("Templates"));
                templateList.add(template);
            }
        }
    } else if ("Layout".equals(String.valueOf(descTemplateType.getType()))) {
    }
}
\end{lstlisting}




\section{Java source files}
Javadoc descriptions are missing for all the classes and methods.

\section{Class and interface declarations}
\subsection{Methods grouping by functionality}
The following method is in the middle of a group of getter methods, grouped by a functionality:
\begin{lstlisting}[firstnumber=330, caption={AttributesEnabled method in the middle of getters}]
public boolean AttributesEnabled() {
    return this.joinedAttrSets != null && this.joinedAttrSets.length > 0;
}
\end{lstlisting}
\section{Method calls}
The following methods are invoked on objects that may be null therefore is needed a check:

\begin{lstlisting}[firstnumber=93, caption={getItemSpecificAttributeSetsForCategories invocation}]
 AttributeSet[] itemSpecAttrSets = attrMaster.getItemSpecificAttributeSetsForCategories(ids);
\end{lstlisting}
\begin{lstlisting}[firstnumber=93, caption={getItemSpecificAttributeSetsForCategories invocation possible solution}]
AttributeSet[] itemSpecAttrSets = null;
if(attrMaster != null){
     itemSpecAttrSets = attrMaster.getItemSpecificAttributeSetsForCategories(ids);
}
\end{lstlisting}
\noindent\makebox[\linewidth]{\rule{\linewidth}{0.4pt}}

\begin{lstlisting}[firstnumber=94, caption={getSiteWideAttributeSetsForCategories invocation}]
AttributeSet[] siteWideAttrSets = attrMaster.getSiteWideAttributeSetsForCategories(ids);
\end{lstlisting}
\begin{lstlisting}[firstnumber=94, caption={getSiteWideAttributeSetsForCategories invocation possible solution}]
AttributeSet[] siteWideAttrSets = null;
if(attrMaster != null){
    siteWideAttrSets = attrMaster.getSiteWideAttributeSetsForCategories(ids);
}
\end{lstlisting}
\noindent\makebox[\linewidth]{\rule{\linewidth}{0.4pt}}

\begin{lstlisting}[firstnumber=95, caption={joinItemSpecificAndSiteWideAttributeSets invocation}]
AttributeSet[] joinedAttrSets = attrMaster.joinItemSpecificAndSiteWideAttributeSets(itemSpecAttrSets, siteWideAttrSets);
\end{lstlisting}
\begin{lstlisting}[firstnumber=95, caption={joinItemSpecificAndSiteWideAttributeSets invocation possible solution}]
AttributeSet[] joinedAttrSets = null;
if(aif != null){
    joinedAttrSets = attrMaster.joinItemSpecificAndSiteWideAttributeSets (itemSpecAttrSets, siteWideAttrSets);
}
\end{lstlisting}
\noindent\makebox[\linewidth]{\rule{\linewidth}{0.4pt}}

\begin{lstlisting}[firstnumber=108, caption={getNameRecommendation invocation}]
this.nameRecommendationTypes = recommendations.getNameRecommendation();
\end{lstlisting}
\begin{lstlisting}[firstnumber=108, caption={getNameRecommendation invocation possible solution}]
if(recommendations != null){
    this.nameRecommendationTypes = recommendations.getNameRecommendation();
}
\end{lstlisting}
\noindent\makebox[\linewidth]{\rule{\linewidth}{0.4pt}}

\begin{lstlisting}[firstnumber=112, caption={get invocation}]
Map<String, CategoryType> categoriesCacheMap = this.siteFacade.getSiteCategoriesMap().get(apiContext.getSite());
\end{lstlisting}
\begin{lstlisting}[firstnumber=112, caption={get invocation possible solution}]
Map<String, CategoryType> categoriesCacheMap = null;
if(this.siteFacade.getSiteCategoriesMap() != null){
    categoriesCacheMap = this.siteFacade.getSiteCategoriesMap().get(apiContext.getSite());
}
\end{lstlisting}
\noindent\makebox[\linewidth]{\rule{\linewidth}{0.4pt}}

\begin{lstlisting}[firstnumber=114, caption={get invocation}]
Map<String, CategoryFeatureType> cfsMap = this.siteFacade.getSiteCategoriesFeaturesMap(). get(apiContext.getSite());
\end{lstlisting}
\begin{lstlisting}[firstnumber=114, caption={get invocation possible solution}]
Map<String, CategoryFeatureType> cfsMap = null;
if(this.siteFacade.getSiteCategoriesFeaturesMap() != null){
    cfsMap = this.siteFacade.getSiteCategoriesFeaturesMap(). get(apiContext.getSite());
}
\end{lstlisting}
\noindent\makebox[\linewidth]{\rule{\linewidth}{0.4pt}}

\begin{lstlisting}[firstnumber=115, caption={get invocation}]
SiteDefaultsType siteDefaults = this.siteFacade.getSiteFeatureDefaultMap().get(apiContext.getSite());
\end{lstlisting}
\begin{lstlisting}[firstnumber=115, caption={get invocation possible solution}]
SiteDefaultsType siteDefaults = null;
if(this.siteFacade.getSiteFeatureDefaultMap() != null){
    siteDefaults = this.siteFacade.getSiteFeatureDefaultMap().get(apiContext.getSite());
}
\end{lstlisting}
\noindent\makebox[\linewidth]{\rule{\linewidth}{0.4pt}}

\begin{lstlisting}[firstnumber=116, caption={get invocation}]
FeatureDefinitionsType featureDefinition = this.siteFacade.getSiteFeatureDefinitionsMap().get(apiContext.getSite());
\end{lstlisting}
\begin{lstlisting}[firstnumber=116, caption={get invocation possible solution}]
FeatureDefinitionsType featureDefinition = null;
if(this.siteFacade.getSiteFeatureDefinitionsMap() != null){
    featureDefinition = this.siteFacade.getSiteFeatureDefinitionsMap(). get(apiContext.getSite());
}
\end{lstlisting}
\noindent\makebox[\linewidth]{\rule{\linewidth}{0.4pt}}

\begin{lstlisting}[firstnumber=121, caption={getItemSpecificsEnabled invocation}]
itemSpecificEnabled = siteDefaults.getItemSpecificsEnabled();
\end{lstlisting}
\begin{lstlisting}[firstnumber=121, caption={getItemSpecificsEnabled invocation possible solution}]
if(siteDefaults != null){
   itemSpecificEnabled = siteDefaults.getItemSpecificsEnabled();
}
\end{lstlisting}
\noindent\makebox[\linewidth]{\rule{\linewidth}{0.4pt}}

\begin{lstlisting}[firstnumber=126, caption={isReturnPolicyEnabled invocation}]
retPolicyEnabled = siteDefaults.isReturnPolicyEnabled();
\end{lstlisting}
\begin{lstlisting}[firstnumber=126, caption={isReturnPolicyEnabled invocation possible solution}]
if(siteDefaults != null){
   retPolicyEnabled = siteDefaults.isReturnPolicyEnabled();
}
\end{lstlisting}
\noindent\makebox[\linewidth]{\rule{\linewidth}{0.4pt}}

\begin{lstlisting}[firstnumber=130, caption={getListingDurations invocation}]
ListingDurationDefinitionsType listDuration = featureDefinition.getListingDurations();
\end{lstlisting}
\begin{lstlisting}[firstnumber=130, caption={getListingDurations invocation possible solution}]
ListingDurationDefinitionsType listDuration = null;
if(featureDefinition != null){
   listDuration = featureDefinition.getListingDurations();
}
\end{lstlisting}
\noindent\makebox[\linewidth]{\rule{\linewidth}{0.4pt}}

\begin{lstlisting}[firstnumber=131, caption={getListingDuration invocation}]
ListingDurationDefinitionType[] durationArray = listDuration.getListingDuration();
\end{lstlisting}
\begin{lstlisting}[firstnumber=131, caption={getListingDuration invocation possible solution}]
ListingDurationDefinitionType[] durationArray;
if(featureDefinition != null){
   durationArray = listDuration.getListingDuration();
}
\end{lstlisting}
\noindent\makebox[\linewidth]{\rule{\linewidth}{0.4pt}}

\begin{lstlisting}[firstnumber=134, caption={getDuration invocation}]
listingDurationMap.put(durationArray[i].getDurationSetID(), durationArray[i].getDuration());
\end{lstlisting}
\begin{lstlisting}[firstnumber=134, caption={getDuration invocation possible solution}]
if(durationArray[i] != null){
    listingDurationMap.put(durationArray[i].getDurationSetID(), durationArray[i].getDuration());
}
\end{lstlisting}
\noindent\makebox[\linewidth]{\rule{\linewidth}{0.4pt}}

\begin{lstlisting}[firstnumber=134, caption={getDurationSetID invocation}]
listingDurationMap.put(durationArray[i].getDurationSetID(), durationArray[i].getDuration());
\end{lstlisting}
\begin{lstlisting}[firstnumber=134, caption={getDurationSetID invocation possible solution}]
if(durationArray[i] != null){
    listingDurationMap.put(durationArray[i].getDurationSetID(), durationArray[i].getDuration());
}
\end{lstlisting}
\noindent\makebox[\linewidth]{\rule{\linewidth}{0.4pt}}

\begin{lstlisting}[firstnumber=134, caption={getDurationSetID invocation}]
listingDurationMap.put(durationArray[i].getDurationSetID(), durationArray[i].getDuration());
\end{lstlisting}
\begin{lstlisting}[firstnumber=134, caption={getDurationSetID invocation possible solution}]
if(durationArray[i] != null){
    listingDurationMap.put(durationArray[i].getDurationSetID(), durationArray[i].getDuration());
}
\end{lstlisting}
\noindent\makebox[\linewidth]{\rule{\linewidth}{0.4pt}}

\begin{lstlisting}[firstnumber=144, caption={getType invocation}]
listingDurationReferenceMap.put(listingDuration[i] .getType().value(),listingDuration[i].getValue());
\end{lstlisting}
\begin{lstlisting}[firstnumber=144, caption={getType invocation possible solution}]
if(listingDuration[i] != null){
    listingDurationReferenceMap.put(listingDuration[i] .getType().value(),listingDuration[i].getValue());
}
\end{lstlisting}
\noindent\makebox[\linewidth]{\rule{\linewidth}{0.4pt}}

\begin{lstlisting}[firstnumber=144, caption={getValue invocation}]
listingDurationReferenceMap.put(listingDuration[i] .getType().value(),listingDuration[i].getValue());
\end{lstlisting}
\begin{lstlisting}[firstnumber=144, caption={getValue invocation possible solution}]
if(listingDuration[i] != null){
    listingDurationReferenceMap.put(listingDuration[i] .getType().value(),listingDuration[i].getValue());
}
\end{lstlisting}
\noindent\makebox[\linewidth]{\rule{\linewidth}{0.4pt}}

\begin{lstlisting}[firstnumber=192, caption={get invocation}]
CategoryType cat = categoriesCacheMap.get(catId);
\end{lstlisting}
\begin{lstlisting}[firstnumber=192, caption={get invocation possible solution}]
CategoryType cat = null;
if(categoriesCacheMap != null){
    cat = categoriesCacheMap.get(catId);
}
\end{lstlisting}
\noindent\makebox[\linewidth]{\rule{\linewidth}{0.4pt}}

\begin{lstlisting}[firstnumber=194, caption={getCategoryLevel invocation}]
if (cat.getCategoryLevel() == 1)
\end{lstlisting}
\begin{lstlisting}[firstnumber=194, caption={getCategoryLevel invocation possible solution}]
if(cat != null && cat.getCategoryLevel() == 1)
\end{lstlisting}
\noindent\makebox[\linewidth]{\rule{\linewidth}{0.4pt}}

\begin{lstlisting}[firstnumber=199, caption={getCategoryParentID invocation}]
return getInheritProperty(cat.getCategoryParentID(0), methodName, categoriesCacheMap, cfsMap);
\end{lstlisting}
\begin{lstlisting}[firstnumber=199, caption={getCategoryParentID invocation possible solution}]
return cat != null ? getInheritProperty(cat.getCategoryParentID(0), methodName, categoriesCacheMap, cfsMap) : null;
\end{lstlisting}
\noindent\makebox[\linewidth]{\rule{\linewidth}{0.4pt}}

\begin{lstlisting}[firstnumber=245, caption={get invocation}]
if (temGroup.get("TemplateGroupId") .equals(descTemplateType.getGroupID().toString()))
\end{lstlisting}
\begin{lstlisting}[firstnumber=245, caption={get invocation possible solution}]
if (temGroup != null && temGroup.get("TemplateGroupId") .equals(descTemplateType.getGroupID().toString()))
\end{lstlisting}
\noindent\makebox[\linewidth]{\rule{\linewidth}{0.4pt}}

\begin{lstlisting}[firstnumber=286, caption={get invocation}]
if (temp.get("TemplateGroupId").equals(temGroupId))
\end{lstlisting}
\begin{lstlisting}[firstnumber=286, caption={get invocation possible solution}]
if (temp != null && temp.get("TemplateGroupId").equals(temGroupId))
\end{lstlisting}
\noindent\makebox[\linewidth]{\rule{\linewidth}{0.4pt}}

\section{Other errors}

The following else if statement should have at least a correct statement to execute.
\begin{lstlisting}[firstnumber=263, caption={statement does nothing}]
} else if ("Layout".equals(String.valueOf(descTemplateType.getType()))) {
}
\end{lstlisting}
\noindent\makebox[\linewidth]{\rule{\linewidth}{0.4pt}}

The following declarations of data structures such as Map and HashMap should have a whitespace after the comma:
\begin{lstlisting}[firstnumber=67, caption={withespace needed after comma}]
private Map<Integer,String[]> listingDurationMap = null;
\end{lstlisting}
\noindent\makebox[\linewidth]{\rule{\linewidth}{0.4pt}}

\begin{lstlisting}[firstnumber=68, caption={withespace needed after comma}]
private Map<String,Integer> listingDurationReferenceMap = null;
\end{lstlisting}
\noindent\makebox[\linewidth]{\rule{\linewidth}{0.4pt}}

\begin{lstlisting}[firstnumber=73, caption={withespace needed after comma}]
private List<Map<String,Object>> adItemTemplates = null;
\end{lstlisting}
\noindent\makebox[\linewidth]{\rule{\linewidth}{0.4pt}}

\begin{lstlisting}[firstnumber=73, caption={withespace needed after comma}]
private List<Map<String,Object>> adItemTemplates = null;
\end{lstlisting}
\noindent\makebox[\linewidth]{\rule{\linewidth}{0.4pt}}

\begin{lstlisting}[firstnumber=221, caption={withespace needed after comma}]
public List<Map<String,Object>> syncAdItemTemplates() throws ApiException, SdkSoapException, SdkException {
\end{lstlisting}
\noindent\makebox[\linewidth]{\rule{\linewidth}{0.4pt}}

\begin{lstlisting}[firstnumber=224, caption={withespace needed after comma}]
List<Map<String,Object>> temGroupList = new LinkedList<Map<String,Object>>();
\end{lstlisting}
\noindent\makebox[\linewidth]{\rule{\linewidth}{0.4pt}}

\begin{lstlisting}[firstnumber=234, caption={withespace needed after comma}]
List<Map<String,Object>> templateList = null;
\end{lstlisting}
\noindent\makebox[\linewidth]{\rule{\linewidth}{0.4pt}}

 \begin{lstlisting}[firstnumber=235, caption={withespace needed after comma}]
Map<String,Object> templateGroup = null;

Map<String,Object> template = new HashMap<String, Object>();
\end{lstlisting}
\noindent\makebox[\linewidth]{\rule{\linewidth}{0.4pt}}


 \begin{lstlisting}[firstnumber=244, caption={withespace needed after comma}]
for (Map<String,Object> temGroup : temGroupList) {
\end{lstlisting}
\noindent\makebox[\linewidth]{\rule{\linewidth}{0.4pt}}

  \begin{lstlisting}[firstnumber=252, caption={withespace needed after comma}]
templateList = new LinkedList<Map<String,Object>>();
\end{lstlisting}
\noindent\makebox[\linewidth]{\rule{\linewidth}{0.4pt}}

  \begin{lstlisting}[firstnumber=252, caption={withespace needed after comma}]
for (Map<String,Object> temGroup : temGroupList) {
\end{lstlisting}
\noindent\makebox[\linewidth]{\rule{\linewidth}{0.4pt}}

\begin{lstlisting}[firstnumber=283, caption={withespace needed after comma}]
public List<Map<String,Object>> getAdItemTemplates(String temGroupId) {
    List<Map<String,Object>> themes = new LinkedList<Map<String,Object>>();
    for (Map<String,Object> temp : this.adItemTemplates) {
\end{lstlisting}
\noindent\makebox[\linewidth]{\rule{\linewidth}{0.4pt}}

\begin{lstlisting}[firstnumber=342, caption={withespace needed after comma}]
public List<Map<String,Object>> getAdItemTemplates() {
\end{lstlisting}
\noindent\makebox[\linewidth]{\rule{\linewidth}{0.4pt}}

The string concatenation should have white spaces before and after the `+` sign.
\begin{lstlisting}[firstnumber=232, caption={withespace needed after comma}]
Debug.logInfo("layout of category "+ this.catId +":"+ resp.getLayoutTotal(), module);
\end{lstlisting}
\noindent\makebox[\linewidth]{\rule{\linewidth}{0.4pt}}
