\chapter{Introduction}

\section{Purpose}
This document represents the Requirement Analysis and Specification Document \(RASD\).
Its aim is to provide a first overview of the system that we want to develop and its main functionalities, in terms of functional requirements, nonfunctional requirements and constraints.
We will integrate these specifications with several diagrams in order to highlight the constraints of our system and its boundaries.

This document is addressed to the developers and programmers, who have to implement it, and to the customers to let them have a first view of the system and the possibility to give us feedbacks according to their needs and opinions.

\section{Actual system}
The company is starting its business right now, so we assume that there is no system providing the requested service.
All the functionalities have to be developed from the beginning.
To develop some functionalities of PowerEnjoy we will rely on external services.

\section{Scope}
The aim of the project PowerEnjoy is to provide an automated service of car sharing.
After a registration, a client can hire a car near him/her through the web or mobile application and he/she can enjoy of all the extra services offered.
The exact position of the client is determined by the GPS signal of the client’s device or it’s allowed to manually insert a specific address.
So the system displays all the available cars in the client’s close area.
Then the client could make a reservation of a car, after which the system notifies the client with a message of confirmation with the car identifier.
If the reservation procedure successfully ends the chosen car won’t be available anymore for other clients.
Moreover a client cannot hire more than one car at the same time.
After the reservation the client has at most one hour to reach the car, when this time expires the system gives a penalty to the client and the car, previously hired, is available again for other clients.
The system allows the client to cancel his/her reservation.
When the client reaches the car, he/she can tell the system that is nearby through a specific button in the application and he/she starts to pay as soon as the engine ignites.
During the travel the system supervises the current charge of the car and notifies it to the client through a screen located in the car.
The system stops charging the amount of money that the client has to pay when he/she communicates through the application his/her decision to stop the rent.
When the car is parked in a safe area and the client exits, the system locks the car automatically and starts the procedure of payment.
The client is notified with the result of this procedure through an SMS, including the final fare.

\section{Actors}
\begin{itemize}
\item \textbf{Guest:} a person that is not already registered in the system or that has to log in.
He/she can only display the home page, with the description of the provided services, and the registration page.
\item \textbf{Registered Client:} a person who has valid access credentials to log in the system (username and password).
Once logged in, he/she can request and hire an available car near him/her, cancel a reservation and he/she have access to all the customer area of the application.
\item \textbf{System Administrators:} a certified user who, after login, has the responsibility to manage administration processes.
He is also in charge of updating data about business logic.

\end{itemize}
\section{Goals}
\begin{enumerate}[label=\textbf{G\arabic*)}]
\item The Registered Client can hire a car through web/mobile application.
\item The Registered Client can hire a car through an SMS.
\item The Guest Client can register himself/herself into the system as Registered Client.
\item Registered Clients can login into the system.
\item The Logged Client can manage his/her sensible data.
\item The Registered Client can manage his/her requests of hiring.
\item Ensure the Registered Client the possibility to receive a discount on his/her last ride.
\item Ensure a uniform distribution of cars in the city.
\item Guarantee to have always a minimum number of cars in the system with enough charge to be hired.
\item Allow the system administrator to update and check data in the database of the system.
\item Guarantee a correct interoperability of the system with external services.
\item Ensure that a Registered Client’s bad behaviour is punished with the application of some penalties.
\end{enumerate}

\section{Definition, acronyms, abbrevations}

\subsection{Definition}

\begin{itemize}
\item \textbf{Guest client:} a person that is not already registered in the system or that has to log in.
\item \textbf{Registered client:} a person who has valid access credentials to log in the system.
\item \textbf{System administrator:} privileged user, in charge of managing administration processes and of updating business logic.
\item \textbf{Reservation:} it is the action performed by a registered client that allow him/her to reserve an available car for maximum one hour.
\item \textbf{Journey time = travel time:} time elapsed since the user starts the engine to the user parks the car and terminates the journey.
\item \textbf{Available car:} a car that is not reserved by any user and has enough charge to be rented.
\item \textbf{Unavailable car:} a car that is already reserved or damaged, so impossible to reserve.
\item \textbf{Gps navigation:} it is the navigation system that is included in the car on board system. It could be used by the user to find direction to the final destination.
\item \textbf{Final destination:} address where the user wants to go.
\item \textbf{Safe area:} the region where is permitted to park and leave a car once the rent is terminated.
\item \textbf{Power grid station:} the area where it’s allowed users to park the cars, leaving them attached to the power grid.
\end{itemize}

\subsection{Acronyms}

\begin{itemize}
\item \textbf{RASD:} Requirements Analysis and Specification Document
\item \textbf{API:} Application Programming Interface
\item \textbf{UI:} User Interface
\end{itemize}

\subsection{Abbrevetations}

\begin{itemize}
\item \textbf{{[Gn]}:} n-goal
\item \textbf{{[Rn]}:} n-functional requirement
\item \textbf{{[An]}:} n-assumption
\end{itemize}

\section{Identify stakeholders}

Our main stakeholder is the owner of PowerEnjoy that wants a state-of-the-art application in order to let users to easily locate, reserve, use and pay a fleet of shared electric cars spread across the city.

\section{Reference documents}

\begin{itemize}
\item PowerEnjoy specification document (assignment).
\item IEEE Std 830-1998 IEEE Recommended Practice for Software Requirements Specifications.
\end{itemize}
\section{Document overview}
The document is substantially divided in 4 sections:
\begin{itemize}
\item Introduction: it gives an high-level description of the software explaining the main purposes, goals and context;
\item Overall description: it provides an overview of the main characteristics of the system to develop, pointing out in particular constraints and assumptions.
\item Specific Requirements: this part contains all the system requirements, typical scenarios and different kind of diagrams useful to understand easily the functionalities.
\item Alloy: the last part of the document contains the description of our model in Alloy, a possible result obtained using this software and the consequent generated world.
\end{itemize}
